\documentclass[twocolumn]{article}
\usepackage{amsfonts}
\usepackage{amssymb}
\usepackage{amsmath}
\usepackage{tikz}
\usetikzlibrary{decorations.pathreplacing}
\usepackage{hyperref}
\hypersetup{
    colorlinks,
    citecolor=black,
    filecolor=black,
    linkcolor=black,
    urlcolor=black
}
\begin{document}

\title{Introduction to Aerospace}
\author{Lecture Notes by Rangi Siebert}
\maketitle
\tableofcontents
\section{The Aviation System}
	\subsection{Evolution of Civil Air Transport}
		\begin{center}
		\begin{tabular}{c|l}
		193? & Junkers F13 \\
		\hline
		1935 & Douglas DC-3 \\
		\hline
		1949/1957 & First flights of the De Havilland Comet\\& and Boeing 707 \\
		\end{tabular}
		\end{center}
		\paragraph{Junkers F13:}
			\begin{itemize}
			\item All Metal
			\end{itemize}
		\paragraph{Milestones of Human Flight: Heavier-than-air-Vessels}
			\begin{itemize}
			\item First flight of the Douglas DC-3 - an \textbf{extremely popular} with airlines; also used for military transport purposes.
			\end{itemize}
		\paragraph{Commercialization of Air Transport (1):}\mbox{}\\
			Civil air transport has developed from an elite way of travel to a \textbf{mass transport system}. The major break threw was with the \textbf{Jet age}.  
		\paragraph{Global Air Traffic:}
			\begin{itemize}
			\item The amount of passengers grow, dew to increased wealth in the middle class
			\end{itemize}	
		\paragraph{Air Traffic Market Outlook: Industry Reports}
			Air traffic is doubling every 15 years, it also contributes 4.3\% of global emissions.
		\paragraph{Overall Demand for New Aircraft and World Fleet Development:}
			When talking about new development, it will take about 30 years, until they will be operation.
		\paragraph{Boundary Conditions - Production Capacities:}

	\subsection{Civil Air Transport: The options}
		\subsubsection{Airline Business Models:}
			\paragraph{Dual-Brand Strategy:}
				\begin{itemize}
				\item Singapore Airlines
				\end{itemize}
			\paragraph{Long-Haul LCC:}
			\paragraph{Hybrid/No-Frills:}

			\paragraph{Full-Service Carrier:}
				\begin{itemize}
				\item Lufthansa
				\item British Airways
				\item Air France
				\item Singapore Airlines
				\item United
				\item ...
				\end{itemize}
			\paragraph{Low-Cost Carrier:}
				\begin{itemize}
				\item Ryanair
				\item Air Asia
				\item ...
				\end{itemize}
				\paragraph{Hub-\&-Spoke VS. Point-to-Point:}
					Passengers do not like Hub, because you have to change, so more passengers are interested in Point-to-Point flights.
				\paragraph{Hub-\&-Spoke Networks:}
					Europe: Main hubs in Paris, Madrid, London, Frankfurt, Rome.\\
					\textbf{Advantages:}
					\begin{itemize}
					\item Reduced costs by using larger aircraft.
					\end{itemize}
				\paragraph{Growth:}
					In Africa, middle East and Asia, the frequency of aircraft movements have increase by about 300\%.
				\paragraph{The Airport:}
					The Airport connects three types of transportation nodes. The space in an airport is quite expensive, so operations have to be highly operational. It also has to be international, so a lot of standardisations have to be considered. (For example the span is limited to 35m). This was OK some years ago, but now this posts a problem. The capacity of airports do not match the demand, this causes longer waiting times, circling above an airport and running engines. This causes 12-13\% more detours being flown. The more detour, the more delays. On the ground: more delays.
		\subsection{Civil Air Transport: The Environmental Impact}
			A concept is the ZEROe from Airbus. However it is not decided jet, what the next generation of aviation will be. Hydrogen is not tested for reliability and safety yes, so it wont be operational before 2050.
			\paragraph{Aviation towards climate neutrality:}
				The goal is to cut down by 50\% by 2050. ${CO_2}$ reduction goals are very ambitious and cannot be met solely by increasing operational efficiency.
				\begin{itemize}%[leftmargin=1em]
				\renewcommand{\labelitemi}{$\Rightarrow$}
				\item Sustainable aviation flues (SAF).
				\end{itemize}
				To be climate neutral, we have to be negative. 

		\subsection{Beyond the Classical Air Transport Market}
			\paragraph{Supersonic Flight:}
				1969 first flight of the Concorde. The highest share of business transit was between North America and Europe. Supersonic flight allowed, flying transatlantic in 3.5 h. The problem is, that the fuel consumption is 500 \% higher.
			\paragraph{Urban Air Mobility:}
				Using major advances of electric motors and batteries \textbf{Air Taxis} could be realised, thaw it comes with an way higher cost in energy. Still it is quite limited dew to its needed infrastructure and costs. \textbf{Vertiports} are small airports dedicated for vertical take-off aircraft. This also has a lot of challenges and an questionable demand. A main challenge is, to get enough renewable energy, to make these concepts possible.
			\paragraph{Topics in Lectures and Research:}
				\begin{itemize}
				\item Airline Fleet Development
				\item Airports
				\item Environmental Aspects
				\item Views of different stakeholders
				\end{itemize}




















\end{document}




















