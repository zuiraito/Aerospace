\documentclass[twocolumn]{article}
\usepackage{amsfonts, amssymb, amsmath}
\usepackage{listings}
\usepackage{tikz}
\usetikzlibrary{decorations.pathreplacing}
\usepackage{graphicx}
\graphicspath{ {./media/} }
\usepackage{hyperref}
\hypersetup{
	colorlinks,
	citecolor=black,
	filecolor=black,
	linkcolor=black,
	urlcolor=black}

\begin{document}
\title{CAD/TB}
\author{Lecture Notes by Rangi Siebert}
\date{\today}
\maketitle
\tableofcontents

\section{Introduction to Technical Drawings}
	\paragraph{The product life cycle \& technical drawings}\mbox{}\\
		\includegraphics[width=\linewidth]{life}
	\paragraph{What is a technical drawing?}\mbox{}\\
		\includegraphics[width=\linewidth]{what}
		\begin{itemize}
		\item The possibility of illustrating forms and transport thoughts in graphic form.
		\item Means of communication between persons and departments.
		\item Form of information and documentation.
		\end{itemize}
		Technical drawings are required to be complete, unambiguous and fully understandable in order to outlive its creator.
	\paragraph{Purpose and classification of sketches:}\mbox{}\\
		\includegraphics[width=\linewidth]{purpose}
	\subsection{Norms and Standards}
		\includegraphics[width=\linewidth]{norms}
		TUM access to a high number of norms through Perinorm: ub.tum.de/perinorm. 
		\subsubsection{DIN EN ISO 5457}
			We will use the drawing sheet according to DIN EN ISO 5457.
		\begin{itemize}
		\item Definition of sheet sizes based on the area of format A0 as 1m$^{2}$.
			\begin{itemize}
			\item[$\rightarrow$] Aspect ration: $\sqrt{2}:1$ 
			\item[$\rightarrow$] Enlargement and reduction: factor $\sqrt{2}$
			\end{itemize}
		\item Creation of next smaller format by halving the area, e.g. $A 0:A 1=2:1$
		\end{itemize}
		\includegraphics[width=\linewidth]{DINA}
		\paragraph{Text field:}
			Completion of a text field for a production drawing with the following information:
			\begin{itemize}
			\item Component designation
			\item Material specification
			\item Tolerance specification
			\item Information of the creator of the drawing
			\item Specification of the scale used
			\end{itemize}
			\includegraphics[width=\linewidth]{textfield}
		\paragraph{Folding:}
			Hand-folding A0 to Form A4 for filing with punched edge:
			\includegraphics[width=\linewidth]{folding}
		\subsubsection{Scales}
			\includegraphics[width=\linewidth]{scales}
			\paragraph{Scales according to DIN ISO 5455:}
				All parts need to be drawn in a defined scale in relation to the original dimensions of the final part:
				\begin{itemize}
				\item Enlargement / reduction scales may be used along defined rations to e.g. fit drawing formats or to highlight small details / enlarge small parts.
				\end{itemize}
				\includegraphics[width=\linewidth]{scales2}
\end{document}
